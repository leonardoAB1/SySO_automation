\resizebox{\textwidth}{!}{
    {\setlength{\fboxsep}{14pt} % Adjust the padding inside the \fbox
    \fbox{
        \begin{tikzpicture}[node distance=2, auto,
            block/.style = {draw, rectangle, fill=blue!10, minimum height=2em, text width=70, align=center},
            arrow/.style = {-{Latex[width=8, length=6]},thick}]
            % Nodes
            \begin{scope}[every node/.style={block}]
            \node (planificacion) {Planificación};
            \node (diseno) [right=of planificacion, xshift=2cm] {Diseño};
            \node (codificacion) [below=of diseno] {Codificación};
            \node (pruebas) [left=of codificacion, xshift=-2cm] {Pruebas};
            \node (lanzamiento) [left=of pruebas] {Lanzamiento};
            \end{scope}

            % Arrows
            \draw[arrow] (planificacion) -- (diseno);
            \draw[arrow] (diseno) -- (codificacion);
            \draw[arrow] (codificacion) -- (pruebas);
            \draw[arrow] (pruebas) -- (planificacion);
            \draw[arrow, dashed] (pruebas) -- (lanzamiento);

            % Annotations
            \node[left=of planificacion, align=right, xshift=1.6cm] {Historias de usuario\\Plan de Iteración\\Criterios de prueba de aceptación};
            \node[right=0.5cm of diseno, align=left] {Diseño Simple\\Tarjetas CRC\\Prototipos};
            \node[right=0.5cm of codificacion, align=left] {Programación\\Rediseño};
            \node[below=0.5cm of pruebas, align=left, xshift=4cm] {Pruebas Unitarias\\Integración Continua};
            \node[below=0.5cm of pruebas, align=left] {Pruebas de\\adaptación};

        \end{tikzpicture}
        }
    }
}