\chapter{Marco teórico}
\label{sec:teoria}
\section{Estado del Arte }
De manera sencilla, el estado del arte puede ser entendido como el conjunto más reciente, o actual, de conocimientos y procedimientos referidos a un tema en específico. Desde otra perspectiva, uno puede indicar que se trata de aquella tecnología de punta que caracteriza al tópico. En ambos casos, la definición acierta en indicar su naturaleza moderna y de innegable trascendencia; el estado del arte marca una tendencia de desarrollo en las comunidades de investigación. 

Un ejemplo, cuando el tópico específico es la “clasificación automática de objetos por medio de computador”, el estado del arte evidente es conformado por los métodos basados en redes neuronales profundas y convolucionales. La razón es clara, estos procedimientos permiten alcanzar los mayores índices de clasificación automática de imágenes. 
Considerando que cada trabajo de grado aborda una temática, es posible determinar que existe un conjunto de conocimientos y procedimientos caracterizados por ser los más actuales, es decir, existe un estado del arte para cada trabajo de grado. En ese sentido, esta sección tiene el propósito de presentar aquellas soluciones previas más recientes que forman parte del estado del arte propio del trabajo. 

La revisión de soluciones previas se distingue por ser un proceso imprescindible dentro la investigación y el diseño ingenieril. Esto se debe a que permite al diseñador, y en este caso estudiante, situarse dentro un espacio de diseño coherente, moderno y significativo. Por tanto, una revisión adecuada debe contemplar múltiples soluciones previas de calidad. 

La cantidad de enfoques es variable, pero al menos son necesarios dos. En ese sentido, se espera que la sección sea desarrollada en al menos dos subsecciones. Por otro lado, con relación a la cantidad de referencias requeridas para considerar que la revisión fue exhaustiva, ésta es de al menos treinta. Esta cantidad corresponde tanto a la perspectiva académica como a la del mercado, siendo al menos diez la cantidad de trabajos académicos.

Claro está, aunque no está demás indicar, que los documentos a referenciar deben ser de alta calidad; i.e. haber sido publicados en sitios cuya política de revisión sea la de pares. Asimismo, los productos del mercado citados deben ser estimados como aquellos que caracterizan la industria o la oferta de máquinas; de distintos proveedores. 
 
Finalmente, antes de pasar a los enfoques, es posible incluir texto de introducción al estado del arte. Esto supone un párrafo que explica la forma en la cual se desarrollará el estado del arte.
%--------------------------------------------------------------------------------------------------------------------
\subsection{SySO en Bolivia}
Históricamente, las primeras disposiciones de orden legal aparecieron a principios del siglo XX. Comenzando con las primeras disposiciones en 1905 sobre pensiones de retiro para maestros hasta la promulgación de la Nueva Ley de Pensiones en 1996, se observa un progreso gradual en la protección de los trabajadores y la promoción de la salud ocupacional en el país.
Se destacan hitos importantes como la creación de la Ley de Accidentes de Trabajo en 1924, que estableció las bases para la indemnización de trabajadores enfermos o accidentados, así como la obligatoriedad de exámenes médicos y normas de higiene y seguridad industrial en todas las industrias.
El establecimiento de la Caja de Seguro y Ahorro Obrero en 1935 y la posterior creación del Ministerio del Trabajo y Previsión Social en 1936 reflejan un mayor compromiso del gobierno con la protección de los derechos laborales. En mayo de 1939 se promulgó la Ley General del Trabajo, que consolidó y organizó todas las disposiciones laborales hasta entonces. Esta ley estableció normas relacionadas con la asistencia médica, la vivienda para los trabajadores, los riesgos laborales y las indemnizaciones, entre otros aspectos. (\cite{cervantesdiagnostico})

A lo largo de las décadas siguientes, se observa un esfuerzo continuo por mejorar las condiciones de trabajo y la atención médica para los trabajadores, con la creación de instituciones como el Instituto Nacional de Salud Ocupacional y la implementación de programas de evaluación médica y seguridad industrial en diversas industrias.
La promulgación de la Ley General de Seguridad Ocupacional y Bienestar en 1979 marcó un hito importante al regular todas las medidas relacionadas con la protección del trabajador en el ambiente laboral.
Finalmente, la reforma del Seguro Social en 1996 representó un paso significativo hacia la modernización del sistema de pensiones y la garantía de la seguridad financiera para los trabajadores en el largo plazo, consolidando así décadas de esfuerzos en el ámbito de la salud ocupacional en Bolivia.
Durante la década del 2000 al 2010, Bolivia experimentó cambios significativos bajo el gobierno de Evo Morales. Se promulgó una nueva Constitución Política del Estado y se propusieron leyes relacionadas con el trabajo y la seguridad social buscando mejorar las condiciones laborales y ampliar la cobertura del sistema de pensiones. Estas acciones reflejaron el compromiso del gobierno por fortalecer los derechos laborales y sociales, así como modernizar la seguridad social para beneficiar a la población trabajadora.
Más aun, \textcite{cervantesdiagnostico} concluye que la mayoría de las leyes y regulaciones existentes en materia de seguridad y salud en el trabajo no se aplican debido a dos razones principales: En primer lugar, debido a obstáculos externos (como condiciones materiales, culturales y de acceso) que dificultan que muchos trabajadores y empleadores cumplan con las normas. En segundo lugar, las entidades encargadas no disponen de las estructuras y recursos necesarios para supervisar su cumplimiento y sancionar las infracciones. Por ejemplo, se señala que si una empresa no cuenta con estímulos del mercado, como una certificación de calidad, es probable que su motivación para cumplir con los estándares de higiene y seguridad sea significativamente menor. Esto se debe a que la supervisión se lleva a cabo únicamente en aquellas empresas que informan accidentes laborales o por el contrario, son denunciadas por conflictos.  

En este contexto, la Resolución Ministerial No. 1411 del 27 de diciembre de 2018 dejó sin efecto los Planes de Higiene, Seguridad Ocupacional y Manual de Primeros Auxilios. Mismos que constituían el mecanismo legal que tenían las empresas para cumplir la Ley General de Higiene y Seguridad Ocupacional y Bienestar, y se aprobó un nuevo mecanismo más eficiente denominado PSST o Programas de Seguridad y Salud en el Trabajo.
Mismo que con la Resolución Ministerial No. 992/23 del 9 de junio del 2023 fue actualizada. La Norma NTS-009/23 define al PGSST o Programa de Gestión de Seguridad y Salud en el Trabajo como: Documento que contiene el conjunto de actividades y mecanismos en materia de higiene, seguridad ocupacional y bienestar implementados en la empresa o establecimiento laboral.
%--------------------------------------------------------------------------------------------------------------------
\section{Fundamentos Teóricos}

% Teoria de Salud y Seguridad
% Teoria Basica de vision
% Teoria Basica de lenguaje natural
