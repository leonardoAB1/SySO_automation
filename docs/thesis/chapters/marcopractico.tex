\chapter{Propuesta de Proyecto}
\label{sec:practico}

% Desarrollo del Proyecto > Por lo menos 50 paginas
% TODO: Requisitos Preliminares > Encuesta a profesionales del area(Apéndice)
% TODO: Análisis Ponderado de Objetivos
% TODO: Análisis/Diagrama Funcional
% TODO: Arquitectura del Programa
% TODO: Casa de la Calidad(Apéndice)
% TODO: Análisis Competitivo
% TODO: Cap xx Análisis de Resultados -> Encuesta para el usuario

En esta sección se describe la propuesta del perfil de tesis/proyecto de grado. Dicha propuesta debe estar basada en el marco teórico, a fin de facilitar la comprensión del lector acerca de la innovación tecnológica que el candidato pretende llevar a cabo durante el desarrollo de su tesis/proyecto de grado.
La propuesta de tesis/proyecto de grado debe hacer uso de diversos recursos, tales como: ecuaciones, diseños CAD, circuitos de diferente índole, tablas, figuras, entre otras que permitan describir con efectividad los objetivos que desea alcanzar en caso de ser aprobado su perfil.
Este apartado o sección, por tanto, es de gran importancia y debe ser detallado con la objetividad y el profesionalismo que se espera de un candidato al grado de ingeniero.

\section{Indice tentativo}
En esta sección se describen de forma detallada todos los capítulos a ser desarrollados en la tesis/proyecto de grado. El número y distribución de los capítulos es de total decisión del autor y su tutor, sin embargo, se sugiere no tener una división mayor a cuatro niveles por capítulo y un número de capítulos no menor a cuatro y no mayor a ocho capítulos.

\section{Análisis económico}
En esta sección se deberá elaborar un análisis económico (en moneda local) preliminar para el desarrollo del proyecto, tomando en cuenta el costo de materiales tanto digitales como físicos necesarios para concluir satisfactoriamente el desarrollo de la Tesis/Proyecto de Grado. 
El desglose económico deberá ser presentado en una tabla con su respectiva subdivisión.

\section{Cronograma de actividades}
En esta sección deberá elaborar un cronograma (basado en Diagrama de Gantt) de todas las actividades que serán desarrolladas para la conclusión satisfactoria del proyecto. 
Este cronograma podrá ser elaborado con el uso de softwares tales como Project o Excel de Microsoft Office, así como por el paquete Pgfgantt de Latex.
Es importante resaltar, que los tiempos y fechas deberán ser coherentes a la complejidad del proyecto y al tiempo de dedicación al mismo. Así también, los periodos de tiempo seleccionados deberán considerar algunas extensiones a fin de permitir subsanar posibles percances durante el desarrollo de experimentos, caso existan.
Las actividades definidas en el diagrama de Gantt, deberán responder a los objetivos específicos planteados.
