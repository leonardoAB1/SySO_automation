\chapter{Entrevista: Ing Paola Aliendre}\label{sec:entrevista01}

\noindent
\textbf{1. ¿Cuál considera usted que es el aspecto más complejo entre los 13 pasos, puntos o documentos que se deben presentar al subir el PGSST a la página? ¿Por qué? ¿O cuál demanda más tiempo para completarse?}\\
De todos los 13 pasos, puntos o documentos que se presentan cuando se sube el PSCT a la página, el más complicado es el punto uno, que describe todas las condiciones de la empresa. Este punto es crucial ya que contiene el soporte técnico de cómo implementar la seguridad, y además se han añadido otros complementos como la gestión de la ergonomía y la parte psicosocial.
\\
\noindent
\textbf{2. ¿Cuánto tiempo requiere la elaboración del punto uno?}\\
En todo el programa de gestión, el punto uno toma alrededor del 30-40\% del tiempo total.
\\
\noindent
\textbf{3. ¿Cuánto tiempo, en términos generales, toma completar todo el programa? Por ejemplo, en días o meses.}\\
Dependiendo del tamaño de la empresa, la implementación del programa completo puede llevar desde tres meses hasta incluso seis meses en el caso de pequeñas empresas sin ningún sistema previo.
\\
\noindent
\textbf{4. ¿Cuáles características le gustaría que el sistema a desarrollar posea?}\\
Se desea un programa que facilite la aplicación de los métodos de gestión de seguridad, especialmente para las pequeñas y medianas empresas, con características que agilicen el proceso y reduzcan los tiempos de desarrollo y de cumplimiento legal.
\\
\noindent
\textbf{5. ¿Podría proporcionarme acceso a una base de datos de proyectos de salud y seguridad previamente realizados para utilizarlos como referencia? ¿O qué alternativas sugiere para obtener recursos similares?}\\
No es posible obtener acceso a una base de datos de proyectos de salud y seguridad ya hechos, pero se sugiere buscar en las bases de datos de las universidades, como la UMSA, donde se pueden encontrar proyectos de grado relacionados con seguridad y salud laboral.
\\
\noindent
\textbf{6. ¿Dónde podría obtener estadísticas referente al PGSST en Bolivia?}\\
La información sobre el sistema de gestión de seguridad se puede encontrar consultando registros en el Ministerio del Trabajo, donde se registran las empresas formales y sus sistemas de gestión.
