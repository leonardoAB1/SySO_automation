\chapter{Introducción}
\label{sec:introduction}

Se sabe que el rendimiento de cualquier empresa está intrínsecamente ligado al factor humano que la conforma (\cite{krekel2019employee}). Por lo tanto, mantener la satisfacción y motivación de los empleados dentro de la organización es un desafío constante que requiere una atención continua. La satisfacción de los trabajadores abarca varios aspectos cruciales para su bienestar y desempeño en el entorno laboral, siendo la seguridad y salud laboral los más prioritarios. Garantizar un ambiente laboral seguro y saludable no solo promueve el bienestar físico y emocional de los empleados, sino que también contribuye significativamente a la productividad y eficiencia general de la empresa. \par
Es responsabilidad de toda organización garantizar la seguridad y salud en el trabajo tanto para sus empleados como para cualquier otra persona que pueda verse afectada por sus operaciones (\cite{Bolivia1979}). La implementación de un Sistema de Gestión de la Salud y Seguridad en el Trabajo (SST) tiene como meta principal proporcionar un entorno laboral seguro y saludable, prevenir lesiones y problemas de salud relacionados con el trabajo, y mejorar continuamente el desempeño en SST.\par
La Norma Técnica de Seguridad (NTS) Nº 009/23, aprobada en Bolivia en 2023, establece las directrices de obligatorio cumplimiento para la presentación y aprobación de los Programas de Gestión de Seguridad y Salud en el Trabajo(PGSST). El PGSST tiene la finalidad de prevenir los riesgos laborales, accidentes de trabajo y enfermedades ocupacionales, a través de la gestión e implementación de mecanismos y medidas en el marco de la normativa legal vigente a fin de reducir y contrarrestar los riesgos al darle prioridad al bienestar del Trabajador ya que es el recurso humano que forma parte de la organización (\cite{cahuasiquitadiseno}).\par
La implementación efectiva del PGSST sigue siendo un desafío debido a varios factores, incluyendo la falta de conciencia sobre la importancia de la SST, la resistencia de algunos sectores empresariales y la falta de recursos para la implementación de las medidas requeridas. Aunque la normativa establece claramente la obligatoriedad de su cumplimiento, muchas empresas aún no se han adaptado al nuevo mecanismo, lo que las expone a multas y sanciones.\par
Además de las estrategias tradicionales para mejorar la seguridad y salud en el trabajo, como la identificación y evaluación de riesgos, capacitaciones, equipos de protección personal (EPP), controles de ingeniería y la promoción de una cultura de seguridad; La implementación de sistemas de gestión de la SST no solo promueve un entorno laboral más seguro y saludable, sino que también impulsa la eficiencia y la efectividad organizacional. Sin embargo, en un mundo cada vez más impulsado por la tecnología, es crucial considerar cómo la Inteligencia Artificial está revolucionando diversas áreas como el desarrollo de software, la atención médica (\cite{Ellahham2020}) o el campo de la Salud y Seguridad Ocupacional (SySO) (\cite{Jarota2023}). La capacidad de la Inteligencia Artificial (IA) para analizar grandes cantidades de datos, identificar patrones y realizar predicciones precisas la convierte en una herramienta invaluable en estos y otros campos, transformando la forma en que se llevan a cabo tareas y se toman decisiones. 
%---------------------------------------------------------------------------------------------
\section{Antecedentes}

Existe un precedente de dificultades históricas y culturales que han afectado la implementación efectiva de medidas de seguridad y salud en el trabajo en Bolivia. \textcite{cervantesdiagnostico} concluye que la mayoría de las leyes y regulaciones existentes en materia de seguridad y salud en el trabajo no se aplican debido a dos razones principales: En primer lugar, debido a obstáculos externos (como condiciones materiales, culturales y de acceso) que dificultan que muchos trabajadores y empleadores cumplan con las normas. En segundo lugar, las entidades encargadas no disponen de las estructuras y recursos necesarios para supervisar su cumplimiento y sancionar las infracciones. Por ejemplo, se señala que si una empresa no cuenta con estímulos del mercado, como una certificación de calidad, es probable que su motivación para cumplir con los estándares de higiene y seguridad sea significativamente menor. Esto se debe a que la supervisión se lleva a cabo únicamente en aquellas empresas que informan accidentes laborales o por el contrario, son denunciadas por conflictos. Por lo cual es crucial abordar estos desafíos con una perspectiva renovada y proactiva.

Para este fin, El Ministerio de Trabajo, a través de la Resolución Ministerial N° 1411/18 de 27 de diciembre de 2018, aprobó la Norma Técnica de Seguridad NTS-009/18 para la presentación y aprobación de los
Programas de Seguridad y Salud en el Trabajo (PSST). Documento de presentación obligatoria, renovado mediante la Resolución Ministerial del MTEPS N°992/23 de 9 de junio de 2023. Desarrollado a fin de posicionar a Bolivia como un país competitivo en materia de salud y seguridad a nivel internacional pues se encuentra basado en la ISO 450001:2018. La primera norma internacional que aborda la seguridad y salud en el trabajo. Diseñada para ayudar a las organizaciones de todos los tamaños y sectores a crear un entorno de trabajo seguro para sus empleados.

Por otra parte, es importante tener en cuenta como se muestra a claramente en la Figura \ref{fig:share_of_industry_exposed_to_automation_ai_gs} cómo la automatización y la inteligencia artificial están transformando rápidamente varios sectores de la industria a nivel mundial. Destacándose el potencial de ser automatizadas las tareas administrativas y legales. De acuerdo a \textcite{hatzius2023potentially}, en estas areas hasta un 46\% de las tareas pueden ser automatizadas. Este cambio tecnológico no solo redefine los roles laborales, sino que también plantea nuevas exigencias. En términos de seguridad y salud ocupacional, la aplicación de estas tecnologías a la realidad nacional tiene el potencial para superar obstáculos externos, así como carencia de recursos y el correspondiente seguimiento que se dificulta en esta región. 

\begin{figure}[htb]
	\centering
	\includegraphics[width=\textwidth]{images/marcoref/share_of_industry_exposed_to_automation_ai.png}
	\caption{Porcentaje de la industria expuesta a la automatización por parte de la IA en Estados Unidos.} \vspace{-0.2cm}
	\footnotesize{{Elaboración propia a partir de ``\textit{\citefield{hatzius2023potentially}{title}}'' (\citeyear{hatzius2023potentially}).}}
	\label{fig:share_of_industry_exposed_to_automation_ai_gs} 
\end{figure}

%---------------------------------------------------------------------------------------------
\section{Planteamiento del Problema}

Como se puede apreciar en la Figura \ref{fig:profesionales_syso_registrados} (a), tras la emisión de la NTS 009/18, vigente a partir de abril del 2019, el numero de profesionales SySO en el Registro Nacional de Profesionales y Técnicos en Higiene, Seguridad Ocupacional y Medicina del Trabajo por mes incrementa rápidamente hasta el periodo de pandemia mundial (\cite{platto2021history}). Después de la cual, vuelve a incrementar el numero de registrados hasta sus máximos históricos en el 2021. A la fecha el numero de registros activos es de 3202 e incrementa a una tasa de crecimiento anual del 59\% tal y como se refleja en la Figura \ref{fig:profesionales_syso_registrados} (b). Por otro lado, según datos oficiales del estatal Servicio Plurinacional de Registro de Comercio, al mes de agosto del 2023, Bolivia registra una base empresarial de 372.969 empresas con una tasa de crecimiento anual de aproximadamente el 4\% (\cite{ministeriodesarrollo2024}). En otras palabras, en un primer momento, se cuenta con 116 empresas por cada profesional SySO registrado. Convirtiendo al PGSST en un negocio altamente atractivo para los profesionales del area, pero no asi en una herramienta que garantice la SySO en todas las empresas.

\begin{figure}[htb]
	\centering
	\includegraphics[width=\textwidth]{images/marcoref/tendencia_profesionales_syso_registrados.png}
	\caption{Registro de profesionales SySO: (a) Mensual (b) Anual.} \vspace{-0.2cm}
	\footnotesize{Elaboración propia a partir de datos del Registro Nacional de Profesionales Técnicos de Higiene, Seguridad Ocupacional y Medicina del Trabajo}
	\label{fig:profesionales_syso_registrados} 
\end{figure}

Nótese que se debe realizar un PGSST no por empresa, sino por establecimiento laboral, independientemente de su tamaño o ubicación. Esto significa que si una empresa tiene varias sucursales o reparticiones, debe presentar un PGSST por cada una de ellas. Si una empresa tiene varias ubicaciones dentro de un mismo establecimiento, debe presentar un PGSST por cada ubicación. La razón de esto es que cada empresa o establecimiento laboral tiene sus propias características y riesgos, por lo que el PGSST debe ser específico para cada uno de ellos. 
Más aun, se debe tomar en cuenta que todos ``los estudios y/o monitoreos deben ser realizados por un profesional inscrito y con credencial categoría `A' vigente en el Registro Nacional'' (\cite{NTS-009/23}). Monitoreos que son necesarios pues se encuentran entre los requisitos principales de un PGSST y deben realizarse de manera regular para garantizar la seguridad y salud de los trabajadores. Sin embargo, como se aprecia en la Figura \ref{fig:profesionales_syso_por_categoria} (a), del numero total de registros activos solo el 56\% es de categoría A; Distribución que como la Figura \ref{fig:profesionales_syso_por_categoria} (b) muestra, se mantiene similar en los diversos departamentos. En otras palabras, por cada 65 empresas hay un profesional activo capacitado para llevar a cabo los estudios de higiene. Por tanto, asumiendo que estos profesionales no trabajan en un mismo equipo, actualmente en el país pueden realizarse 1792 estudios de higiene al mismo tiempo; En caso el estudio de higiene de una empresa tome un solo dia y que la misma cuente con solo una ubicación, se necesitan 208 días para que los profesionales realicen o actualicen a todas las empresas. En efecto, algunos estudios, como los de ruido o iluminación, pueden realizarse en un solo día, pero hay otros, como los de calidad del aire o de agentes químicos, pueden requerir varios días o incluso semanas.  

\begin{figure}[htb]
	\centering
	\includegraphics[width=\textwidth]{images/marcoref/profesionales_syso_por_categoria.png}
	\caption{Profesionales SST activos: Distribución de categorías (a) porcentual (b) por departamento.} \vspace{-0.2cm}
	\footnotesize{Elaboración propia a partir de datos del Ministerio Trabajo, Empleo y Previsión Social}
	\label{fig:profesionales_syso_por_categoria} 
\end{figure}

En resumen, la actual normativa vigente, aunque ha generado un incremento significativo en el registro de profesionales SySO y en la atención a la seguridad laboral, enfrenta desafíos que limitan su efectividad. La complejidad en la generación de PGSST, las deficiencias en capacitación y los desafíos de cumplimiento normativo constituyen un problema multifacético que compromete tanto la seguridad de los empleados como el rendimiento operativo de las empresas. Es evidente que se requieren acciones urgentes para abordar estas dificultades y garantizar entornos laborales seguros y productivos para el creciente número de empresas en el país.

%---------------------------------------------------------------------------------------------
\section{Definición del Problema} 
¿Cómo impacta la identificación de riesgos laborales en la gestión eficaz de los Programas de Gestión de Seguridad y Salud en el Trabajo (PGSST)?

%-----------------------------------------------------------------------------------------------------------------------------
\section{Objetivos}
\subsection{Objetivo general}
Implementar un sistema automatizado tomando de base la NTS-009/23 que permita la identificación de riesgos laborales, facilitando el desarrollo y la gestión eficaz de los PGSST.

\subsection{Objetivos específicos}
\begin{itemize}
	\item Entrevistar a las partes interesadas clave, incluyendo profesionales SySO y usuarios potenciales del sistema, para comprender completamente sus necesidades, requisitos y expectativas.
	\item Analizar la situación actual de los proceso que se llevan a cabo actualmente para desarrollar los documentos requeridos por la NTS-009/23, tomando en consideración los estándares de la ANSI para la elaboración de diagramas de flujo.
	\item Establecer la arquitectura del sistema, incluyendo la estructura general, los componentes principales y las tecnologías a utilizar, para asegurar la escalabilidad, la modularidad y la eficiencia del sistema por medio del anal.
	\item Elaborar un cronograma del proyecto detallando cada iteración del proceso de desarrollo por medio de la técnica de priorización MoSCoW para asegurar un Producto Mínimo Viable (MVP por sus siglas en inglés) en la menor cantidad de tiempo.
	\item Desarrollar las iteraciones del proyecto, estableciendo los objetivos específicos a alcanzar durante la fase de diseño, implementación y pruebas del sistema.
	\item Implementar el sistema utilizando la metodología de Desarrollo Guiado por Pruebas (TDD por sus siglas en inglés), donde se escriben pruebas unitarias antes de desarrollar el código, asegurando así la calidad y la robustez del sistema.
	\item Evaluar la calidad del software tras cada iteración como parte del proceso de pruebas del sistema haciendo uso de la metodología Goal-Question-Metric (Objetivo-Pregunta-Métrica) para asegurar la calidad en base al estándar ISO 25000:2014.
\end{itemize}

%-----------------------------------------------------------------------------------------------------------------------------
\section{Justificación}
\subsection{Justificación legal}
La justificación legal de este proyecto se fundamenta en el marco normativo vigente en Bolivia, específicamente en la NTS-009/23, la Ley General de Higiene, Seguridad Ocupacional y Bienestar (D.L. 16998 del 2 de agosto de 1979), Reglamento de la Ley general del trabajo (Decreto Supremo Nº 224 de 23 de agosto de 1943) y otras regulaciones relacionadas.

En primer lugar, la NTS-009/23 establece claramente los requisitos que deben cumplir todas las empresas y lugares de trabajo, tanto nacionales como extranjeros, en relación con la seguridad y salud ocupacional. Esta normativa especifica la obligación de desarrollar y mantener Programas de Gestión de Salud y Seguridad en el Trabajo (PGSST) que garanticen un entorno laboral seguro y saludable para todos los trabajadores. Por lo tanto, la implementación de un sistema automatizado para la gestión de la seguridad y salud ocupacional se justifica en el contexto de cumplimiento de esta normativa, ya que facilitará el cumplimiento de los requisitos establecidos y mejorará la eficiencia en la gestión de los PGSST. En este sentido, la implementación de un sistema automatizado para la gestión de la seguridad y salud ocupacional se alinea con los principios y objetivos establecidos en esta la Ley General de Higiene, Seguridad Ocupacional y Bienestar, contribuyendo a fortalecer el cumplimiento de las disposiciones legales y a proteger los derechos y la integridad de los trabajadores en el país.

En conclusión, la justificación legal de este proyecto se sustenta en la necesidad de cumplir con la normativa vigente en materia de seguridad y salud ocupacional en Bolivia. La implementación de un sistema automatizado para la gestión de la seguridad y salud ocupacional ayudará a garantizar el cumplimiento de los requisitos legales establecidos, promoverá un entorno laboral seguro y saludable, y protegerá los derechos y el bienestar de los trabajadores en el país.

\subsection{Justificación tecnológica}

La justificación tecnológica de este proyecto radica en la necesidad de aprovechar las ventajas de la Inteligencia Artificial (IA) y las tecnologías de información para mejorar la eficiencia y la efectividad en la gestión de la seguridad y salud ocupacional en las empresas bolivianas.

La implementación de un sistema automatizado basado en IA permitirá identificar proactivamente riesgos laborales, agilizar el desarrollo de Programas de Gestión de Salud y Seguridad en el Trabajo (PGSST) y facilitar la gestión eficiente de los mismos. Esta tecnología proporcionará capacidades avanzadas para analizar grandes cantidades de datos, identificar patrones y tendencias, y realizar predicciones precisas, lo que permitirá una toma de decisiones más informada y rápida en materia de seguridad y salud ocupacional.

En resumen, la adopción de tecnologías de IA y automatización en la gestión de la seguridad y salud ocupacional ofrece una oportunidad única para mejorar la seguridad de los trabajadores, reducir los riesgos laborales y promover un entorno laboral más seguro y saludable en las empresas bolivianas. Esta justificación tecnológica respalda la necesidad de desarrollar e implementar un sistema automatizado para la gestión de la seguridad y salud ocupacional en el contexto específico de Bolivia.

\subsection{Justificación social}

La justificación social de este proyecto se basa en el impacto positivo que tendrá en la sociedad boliviana al mejorar las condiciones de trabajo y promover un entorno laboral seguro y saludable para todos los trabajadores.

La implementación de un sistema automatizado para la gestión de la seguridad y salud ocupacional beneficiará directamente a los trabajadores al reducir los riesgos laborales y prevenir accidentes y enfermedades relacionadas con el trabajo en menor tiempo y a menor costo. Por ejemplo, según entrevistas realizadas a consultores SySO el tiempo promedio que se tarda en desarrollar un PGSST para Pequeñas y Medianas Empresas (PyME) mediante las metodologías tradicionales es de seis meses (Ver Anexo A para más detalles). La facilidad que se otorgará a disminuir el tiempo de desarrollo, se traduce en un menor precio lo que hace que este sistema sea más accesible para las empresas de todo tipo. Además, un entorno laboral seguro y saludable también tendrá un impacto positivo en la productividad y la eficiencia de las empresas, lo que a su vez beneficiará a la economía en general. Menos accidentes y enfermedades laborales significan menos días de trabajo perdidos, menores costos de atención médica y una fuerza laboral más comprometida y motivada.

En última instancia, la implementación de este sistema contribuirá a construir una cultura de seguridad y salud ocupacional en Bolivia, donde la protección de los trabajadores y el cumplimiento de las normativas de seguridad laboral sean prioridades fundamentales para todas las empresas y organizaciones. Esto no solo beneficiará a los trabajadores individuales y a las empresas, sino que también contribuirá al desarrollo sostenible y al bienestar general de la sociedad boliviana.

%-----------------------------------------------------------------------------------------------------------------------------

\section{Delimitación}
\subsection{Limites}

\noindent
Los límites del proyecto se describen a continuación:

\begin{itemize}[topsep=0pt]
	\item El contenido de los documentos generados por el software se encuentra delimitado por los requerimientos técnicos y correspondientes entregables expuestos en NTS-009/23. 
	\item El contenido generado por el software deberá estar en conformidad con la norma ISO 45001:2018, la norma internacional para sistemas de gestión de seguridad y salud en el trabajo (SGSST).
	\item La ISO 25000:2014 proporciona un marco para la evaluación de la calidad del producto de software. Se utilizarán sus directrices para asegurar que el software desarrollado cumpla con los estándares de calidad y satisfaga las necesidades de los usuarios finales.
	%\item El sistema se deberá adherir al estándar ISO 420001:2023, mismo que proporciona un marco para gestionar los riesgos y oportunidades asociados con la IA, para garantizar la calidad y la seguridad en todas las operaciones de IA.
	\item Para utilizar la aplicación se tendrá que estar conectado a internet. 
	En caso que no se esté conectado a internet, no se tendrá acceso a la aplicación.
	\item El software no contará con herramientas especializadas en la automatización de los estudios de higiene. Se limitará a plantillas para rellenar y automatizar el análisis de los datos recolectados por ``un profesional inscrito y con credencial categoría ``A'' vigente en el Registro Nacional de Profesionales Técnicos de Higiene, Seguridad Ocupacional y Medicina del Trabajo'' (\cite{NTS-009/23}).
\end{itemize}

\subsection{Alcances}

\noindent
El alcance del proyecto incluirá:

\begin{itemize}
	\item Prototipo funcional de aplicación web que permita agilizar el proceso de desarrollo un de PGSST mediante generación automática de por lo menos tres de los trece documentos solicitados por la NTS 009/23 de acuerdo a los requisitos recolectados de entrevistas a profesionales en el area.
	\item Entrevistas realizadas a los usuarios objetivo del producto.
	\item Documentación de la investigación de las tecnologías que se utilizarán para el
	desarrollo del software final.
	\item Documentación del desarrollo del software, en el cual se incluirá los detalles de la automatización del desarrollo de un PGSST y un manual de usuario.
\end{itemize}

\noindent
Las exclusiones del proyecto son las siguientes:
\begin{itemize}
	\item No se implementarán los módulos, audio, inteligencia artificial, interfaz de usuario y entrada desde cero, se utilizará bibliotecas open-source disponibles.
	\item El proyecto no estará dirigido a empresas o rubros que requieran de normativa o procedimientos particulares, como es el caso de la industria petrolera.
\end{itemize}

Tal y como indica norma, todo PGSST desarrollado, debe ser aprobado por personal debidamente inscrito y con credencial vigente en el Registro Nacional de Profesionales y Técnicos en Higiene, Seguridad Ocupacional y Medicina del Trabajo, a cargo del Ministerio Trabajo, Empleo y Previsión Social (MTEPS) conforme a normativa legal vigente (\cite{NTS-009/23}).
